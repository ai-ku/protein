\documentclass[12pt,article]{iopart}


\begin{document}

\noindent We thank all four reviewers for their valuable comments and
criticisms. We have improved our manuscript along the lines pointed
out, modifying the parts that were unclear and adding further content
where necessary. Below is our detailed response to each point that was
raised, together with a list of the changes in the manuscript. We hope
that our submission in its present form is sufficiently clear and
accurate to merit publication in Physical Biology.

\section{Response to the comments of the first referee:}

 
The referee liked the use of Hermite polynomials for the PDF that
includes mode coupling and anharmonicity, but raised 3 points which we
try to clarify below: \\ \\
{\it Firstly, the anharmonicity and mode-coupling
contributions should be carefully observed via the free energy
surfaces for various kinds of modes, such as
``anharmonic'', ``quasiharmonic'', and ``harmonic'' modes, without being
limited to only mode 1 and 2. Then the entropy should be calculated
separately. For ``anharmonic'' and ``quasiharmonic'' modes the resulting
entropy may be dominated by the anharmonicity and the mode‑coupling.}\\ \\
We tried to clarify the role of various modes in our approach in the
new paragraph that we added to the end of the Conclusion
section (please see our response to the second referee).\\ \\
{\it Secondly, I doubt whether the MD simulation time (here, 900 ps) will
be enough for estimating the higher-order polynomials. Much longer
simulation should be performed to examine whether the derived
polynomials are well converged.} \\ \\
{\bf Bunu Deniz yapacak.}\\ \\
{\it Thirdly, it is not clear how the Hermite polynomials are
  calculated by eq. 5. The harmonic approximation is applied for the
  PDF in eq. 5? And how many terms (i.e., rank) are summed up for
  deriving f1 and f2?}\\ \\
Eq.(5) demonstrates that the coefficients to be determined are the
expectation values of the various Hermite (bi)polynomials. In order to
clarify how these expected values are estimated, we have added a paragraph to
the end of Section 3.1 which reads:

``The average values of the Hermite polynomials given in Eq.(5) are
calculated by averaging over 7971 snapshots of the fluctuation
trajectory. A maximum rank of 17 for the Hermite polynomial tensors
was observed to be sufficient for the convergence of our results.''

\section{Response to the comments of the second referee:}

The referee recommended the publication of the manuscript in Physical
Biology, but raised a point which we try to clarify below: \\ \\
{\it There are other approaches used to compute free energy landscapes
that naturally incorporate anharmonicity, specifically free energy
landscapes determined in terms of lowest principal components or
dihedral principal components, along which the protein makes
relatively large displacements. This has been done, for example, by
Stock and coworkers (J. Chem. Phys. 128, 245102, (2008)) and
Maisuradze and Leitner (Proteins: Structure, Function and
Bioinformatics 67, (2007)) and Maisuradze et al (Phys. Rev. Lett. 102
238102 (2009)). It would be useful if the authors would compare and
discuss the merits of their approach to determine free energy
landscape of a protein near its native structure against other
approaches such as these.}\\ \\
%
We thank the reviewer for pointing out the three references that bear
resemblance to ours in that each treat large scale motions of the
protein. We now included the following explanations and comparisons at
the end of the Conclusion section and the papers in the References:

``Anharmonicity of protein motions has been addressed in several
earlier papers by means of principal component analysis
[3,15-17]. Hayward {\it et al.} [3] used a quasiharmonic approximation
to the anharmonic energy surface having multiple minima, where the
fluctuation distribution function was treated as a multivariable
Gaussian, with the variables being the normal modes of the molecular
dynamics trajectory. If hopping between different minima is
suppressed, then the quasiharmonic approximation reduces to normal
mode analysis. In the present paper, the quasiharmonic approximation
would obtain if the sum in the brackets in Eq.(2) were set to zero,
subject to the conditions given by Eqs.(1\&3).

The higher-order moments in Eq.(2) include terms necessary to go
beyond the quasiharmonic approximation. Mausiradze and Leitner [15]
applied the principal component analysis to a tetrapeptide and
analyzed the free energy surface using only the largest two
eigenvectors obtained from the dihedral angle space. Their analysis
contains the coupling effects between the first two eigenvectors. In
the present work we also expressed the free energy surface keeping
mode pairs, but contrary to the the treatment of Ref.[15], all pairs
are included. Ref.[15] also discussed the problems with sampling
convergence based on the first two mode analysis. In our case {\bf
  …(Denizin dun anlattiklarina uyan birsey ekeleyebilir miyiz buraya,
  veya Denizin argumanlarini buraya koyabiliriz.}

Maisuradze {\it et al.} [16] investigated the folding and unfolding of
the B domain of staphylococcal protein by a coarse-grained principal
component analysis and showed that while a one or two dimensional free
energy landscape is sufficient for describing folding and unfolding,
it may fail in describing the stability of the native state. Their
work considers the folding/unfolding of a protein, and therefore
deviates from the present work that focuses on fluctuations about the
native state. Nevertheless, their observations are relevant, and the
need to go to higher dimensions is pertinent. Similarly, Altis {\it et
  al.}  showed that a five dimensional landscape in dihedral space is
necessary to properly characterize the free energy landscape [17].

The main difference of the present work from past studies above is
that, by means of a Hermite expansion we take into account all of the
principal components, in contrast with the restricted numbers
considered in [15-17]. The constraint on the Hermite series expansion
is twofold: maximum tensor rank and the order of coupling. A cutoff on
the maximum rank can be made as large as computationally possible,
since their evaluation is straightforward. The order of coupling
(two-body, three-body, etc) is more subtle. In the present paper, we
treated the first- and second-order couplings only. Including the
higher-order terms is a computational challange. Instead, we
supplemented our results by an alternative nonparametric ``kernel
density'' estimation method that in effect considers all orders, but
does not allow a separation of elastic, anharmonic and mode-coupling
contributions to the free energy. Our results show that the difference
between the Hermite representation at the second-order and the KDE do
not show marked differences from each other.''\\ \\ 
%
We added the following references in the revised version:\newline [15]
Maisuradze and Leitner (Proteins: Structure, Function and
Bioinformatics 67, (2007))\newline [16] Maisuradze {\it et al.}
(Phys. Rev. Lett. 102 238102 (2009))\newline [17] A. Altis, M. Otten,
P. H. Nguyen, R. Hegger, G. Stock (J. Chem. Phys. 128, 245102, (2008))

\section{Response to the comments of the third referee}

The referee strongly supported the publication of this paper to the Physical
Biology without any comments for clarification.


\section{Response to the comments of the fourth referee}

The referee found the manuscript suitable for publication in Physical
Biology. He added three comments to which we respond below: \\ \\ 
%
{\it The authors define initially fluctuations as deviations from
  equilibrium positions, but then use in practice average atom
  position in the simulation as an estimate of the former. I would
  argue that the latter is in fact the correct definition, as it makes
  possible the analysis of ``double well'' cases, such as the one
  found for crambin (see fig. 3). Indeed, in case of asymmetric wells,
  average and equilibrium position do not correspond.}\\ \\
%
We thank the referee and agree with her/him that the the proper
expression should have been the ``mean positions'' of the atoms,
rather than ``equilibrium''. This has now been corrected accordingly
in the text and the superscripts have been changed from $R^{eq}$ to
$R^{0}$ in order to avoid confusion. \\ \\
%
{\it The authors find that higher order computation of non-harmonic
  corrections increases the fluctuation entropy. Is this a general
  rule?  Could this be related to the ``double well'' feature of the
  slowest mode?} \\ \\ 
%
{\bf This, in fact, is provably so, since .. (Deniz). We have now
  clarified this point also in the manuscript..}\\ \\ 
%
{\it The eigenvalues of the covariance matrix are just mentioned and
  never appear in any equation. It could be useful for the general
  reader to show how they actually enter the game (e.g. in the
  evaluation of entropy for the pure harmonic case). Moreover: could
  one state that mode coupling is relevant for mode pairs chosen among
  the slowest modes?}\\ \\
%
The normalization of the normal modes' standard deviation to unity is
done by stretching the normal axes by an amount proportional to the
inverse square-root of the eigenvalues. From this point on, they are
invisible in our formulation except for determining the ordering of
the modes: the larger a mode's eigenvalue (the smaller the spring
constant) the smaller is the rank. This is now stated clearly in the
sentence after Eq.(3).

As for the fluctuation entropy, note that in a harmonic system
($f_0$), each mode contributes equally to the free energy irrespective
of its eigenvalue. The referee's comment is valuable, since it
stresses that the different contributions from different eigenvalues
is a direct consequence of non-harmonic effects. In fact, the
anharmonic contributions are most prominant for the modes with the
largest eigenvalues, however finding a quantitative relationship
appears to be a highly nontrivial task. We now make this explicit at
the end of the first paragraph in the Results section.



\end{document}
